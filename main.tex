% The formatting of this CV is based on @davidwhogg's layout.

\documentclass[12pt,letterpaper]{article}

\usepackage{color}
\usepackage{fancyhdr}
\usepackage{hyperref}
\usepackage{ifthen}

% \usepackage[yyyymmdd]{datetime}
% \renewcommand{\dateseparator}{-}

% Link formatting.
\definecolor{numcolor}{rgb}{0.5,0.5,0.5}
\definecolor{linkcolor}{rgb}{0,0,0.4}
\hypersetup{%
    colorlinks=true,        % false: boxed links; true: colored links
    linkcolor=linkcolor,    % color of internal links
    citecolor=linkcolor,    % color of links to bibliography
    filecolor=linkcolor,    % color of file links
    urlcolor=linkcolor      % color of external links
}

% Text formatting.
\newcommand{\foreign}[1]{\textit{#1}}
\newcommand{\etal}{\foreign{et~al.}}
\newcommand{\project}[1]{\textsl{#1}}
\definecolor{grey}{rgb}{0.5,0.5,0.5}
\newcommand{\deemph}[1]{\textcolor{grey}{\footnotesize{#1}}}

% literature links--use doi if you can
  \newcommand{\doi}[2]{\emph{\href{http://dx.doi.org/#1}{{#2}}}}
  \newcommand{\ads}[2]{\href{http://adsabs.harvard.edu/abs/#1}{{#2}}}
  \newcommand{\isbn}[1]{{\footnotesize(\textsc{isbn:}{#1})}}
  \newcommand{\arxiv}[1]{{\href{http://arxiv.org/abs/#1}{arXiv:{#1}}}}

% Section headings.
\newcommand{\cvheading}[1]{\addvspace{1ex}\pagebreak[2]\par\textbf{#1}\nopagebreak\vspace{-0.4em}}

% Set up the custom unordered list.
\newcounter{refpubnum}
\newcommand{\cvlist}{%
    \rightmargin=0in
    \leftmargin=0.15in
    \topsep=0ex
    \partopsep=0pt
    \itemsep=0.2ex
    \parsep=0pt
    \itemindent=-1.0\leftmargin
    \listparindent=0.0\leftmargin
    \settowidth{\labelsep}{~}
    \usecounter{refpubnum}
}

% Margins and spaces.
\raggedright
\setlength{\oddsidemargin}{0in}
\setlength{\topmargin}{0in}
\setlength{\headsep}{0.20in}
\setlength{\headheight}{0.25in}
\setlength{\textheight}{9.1in}
\addtolength{\topmargin}{-\headsep}
\addtolength{\topmargin}{-\headheight}
\setlength{\textwidth}{6.50in}
\setlength{\parindent}{0in}
\setlength{\parskip}{1ex}

% Headings and footings.
\renewcommand{\headrulewidth}{0pt}
\pagestyle{fancy}
\lhead{\deemph{Arjun Baliga Savel}}
\chead{\deemph{Curriculum Vitae}}
\rhead{\deemph{\thepage}}
\cfoot{\deemph{\textit{Last updated: \today}}}

% Journal names.
\newcommand{\aj}{AJ}
\newcommand{\apj}{ApJ}
\newcommand{\pasp}{PASP}
\newcommand{\mnras}{MNRAS}

\begin{document}\thispagestyle{empty}\sloppy\sloppypar\raggedbottom

\textbf{\Large Arjun Baliga Savel} \hfill
\textsf{\small asavel@gmail.com, https://arjunsavel.github.io} \\

\cvheading{Education}
\begin{list}{}{\cvlist}
\item
\textbf{Ph.D., Astronomy}, University of Maryland (expected)
\item
\textbf{M.S., Astronomy}, University of Maryland (expected)
\item
\textbf{B.A., Astrophysics; B.A., Physics}, University of California, Berkeley (2020) \\\textit{Advisor:} Courtney Dressing
\end{list}

\cvheading{Research interests}
\begin{list}{}{\cvlist}

\item Extracting information from exoplanet atmospheres
\item Characterizing exoplanetary systems
\item Ststistically constraining exoplanet properties, especially with respect to habitability
\item Constructing rigorously motivated machine learning applications to astronomy

\end{list}

\cvheading{Selected Honors, Prizes, and Awards}
\begin{list}{}{\cvlist}

\item Gregor and Donat Wentzel Scholarship, University of Maryland (2020)
\item Student commencement speaker, UC Berkeley Astronomy Department (2020)
\item $^\ddagger$Chambliss Astronomy Achievement Award Student Prize, AAS 235 (2020)
\item $^\dagger$Outstanding Graduate Student Instructor Award, UC Berkeley (2020)
\item $^*$1st place at Astronomy Poster Summer Intern Symposium (APSIS), UC Berkeley (2019)
\item Student Technology Fund grant for ULAB, UC Berkeley (2018)
\item Ongoing UC Berkeley Physics Department award for ULAB (2018)

\end{list}


\ifdefined\withpubs
    \cvheading{Publications}
    citations: 114 / h-index: 6 (2022-05-10)

    \cvheading{Refereed publications}
    \begin{list}{}{\cvlist}
    \item[{\color{numcolor}\scriptsize30}] Parviainen, Hannu; Murgas, Felipe; Esparza-Borges, Emma; Pel{\'a}ez-Torres, A. \etal\ ({59} other co-authors, incl.\ \textbf{Savel, Arjun}) 2024, \doi{10.48550/arXiv.2401.11879}{TOI-2266 b: a keystone super-Earth at the edge of the M dwarf radius valley}, ArXiv (\arxiv{2401.11879})

\item[{\color{numcolor}\scriptsize29}] Malsky, Isaac; Rauscher, Emily; Roman, Michael T.; Lee, Elspeth K. H. \etal\ ({5} other co-authors, incl.\ \textbf{Savel, Arjun}) 2024, \doi{10.3847/1538-4357/ad0b70}{A Direct Comparison between the Use of Double Gray and Multiwavelength Radiative Transfer in a General Circulation Model with and without Radiatively Active Clouds}, \apj, {961}, 66 (\arxiv{2311.01506})

\item[{\color{numcolor}\scriptsize28}] Rasmussen, Kaitlin C.; Currie, Miles H.; Hagee, Celeste; van Buchem, Christiaan \etal\ ({17} other co-authors, incl.\ \textbf{Savel, Arjun}) 2023, \doi{10.3847/1538-3881/acf28e}{A Nondetection of Iron in the First High-resolution Emission Study of the Lava Planet 55 Cnc e}, \aj, {166}, 155 (\arxiv{2308.10378}) [\href{https://ui.adsabs.harvard.edu/abs/2023AJ....166..155R}{1 citation}]

\item[{\color{numcolor}\scriptsize27}] Coulombe, Louis-Philippe; Benneke, Bj{\"o}rn; Challener, Ryan; Piette, Anjali A. A. \etal\ ({73} other co-authors, incl.\ \textbf{Savel, Arjun}) 2023, \doi{10.1038/s41586-023-06230-1}{A broadband thermal emission spectrum of the ultra-hot Jupiter WASP-18b}, Nature, {620}, 292 (\arxiv{2301.08192}) [\href{https://ui.adsabs.harvard.edu/abs/2023Natur.620..292C}{28 citations}]

\item[{\color{numcolor}\scriptsize26}] Kempton, Eliza M. -R.; Zhang, Michael; Bean, Jacob L.; Steinrueck, Maria E. \etal\ ({30} other co-authors, incl.\ \textbf{Savel, Arjun}) 2023, \doi{10.1038/s41586-023-06159-5}{A reflective, metal-rich atmosphere for GJ 1214b from its JWST phase curve}, Nature, {620}, 67 (\arxiv{2305.06240}) [\href{https://ui.adsabs.harvard.edu/abs/2023Natur.620...67K}{22 citations}]

\item[{\color{numcolor}\scriptsize25}] Tuson, A.; Queloz, D.; Osborn, H. P.; Wilson, T. G. \etal\ ({119} other co-authors, incl.\ \textbf{Savel, Arjun}) 2023, \doi{10.1093/mnras/stad1369}{TESS and CHEOPS discover two warm sub-Neptunes transiting the bright K-dwarf HD 15906}, \mnras, {523}, 3090 (\arxiv{2306.04511}) [\href{https://ui.adsabs.harvard.edu/abs/2023MNRAS.523.3090T}{7 citations}]

\item[{\color{numcolor}\scriptsize24}] Dai, Fei; Schlaufman, Kevin C.; Reggiani, Henrique; Bouma, Luke \etal\ ({48} other co-authors, incl.\ \textbf{Savel, Arjun}) 2023, \doi{10.3847/1538-3881/acdee8}{A Mini-Neptune Orbiting the Metal-poor K Dwarf BD+29 2654}, \aj, {166}, 49 (\arxiv{2306.08179}) [\href{https://ui.adsabs.harvard.edu/abs/2023AJ....166...49D}{3 citations}]

\item[{\color{numcolor}\scriptsize23}] Gao, Peter; Piette, Anjali A. A.; Steinrueck, Maria E.; Nixon, Matthew C. \etal\ ({12} other co-authors, incl.\ \textbf{Savel, Arjun}) 2023, \doi{10.3847/1538-4357/acd16f}{The Hazy and Metal-rich Atmosphere of GJ 1214 b Constrained by Near- and Mid-infrared Transmission Spectroscopy}, \apj, {951}, 96 (\arxiv{2305.05697}) [\href{https://ui.adsabs.harvard.edu/abs/2023ApJ...951...96G}{9 citations}]

\item[{\color{numcolor}\scriptsize22}] Beltz, Hayley; Rauscher, Emily; Kempton, Eliza M. -R.; Malsky, Isaac \etal\ ({2} other co-authors, incl.\ \textbf{Savel, Arjun}) 2023, \doi{10.3847/1538-3881/acd24d}{Magnetic Effects and 3D Structure in Theoretical High-resolution Transmission Spectra of Ultrahot Jupiters: the Case of WASP-76b}, \aj, {165}, 257 (\arxiv{2302.13969}) [\href{https://ui.adsabs.harvard.edu/abs/2023AJ....165..257B}{3 citations}]

\item[{\color{numcolor}\scriptsize21}] Rodriguez, Joseph E.; Quinn, Samuel N.; Vanderburg, Andrew; Zhou, George \etal\ ({130} other co-authors, incl.\ \textbf{Savel, Arjun}) 2023, \doi{10.1093/mnras/stad595}{Another shipment of six short-period giant planets from TESS}, \mnras, {521}, 2765 (\arxiv{2205.05709}) [\href{https://ui.adsabs.harvard.edu/abs/2023MNRAS.521.2765R}{12 citations}]

\item[{\color{numcolor}\scriptsize20}] \textbf{Savel}, \textbf{Arjun}; Kempton, Eliza M. -R.; Rauscher, Emily; Komacek, Thaddeus D. \etal 2023, \doi{10.3847/1538-4357/acb141}{Diagnosing Limb Asymmetries in Hot and Ultrahot Jupiters with High-resolution Transmission Spectroscopy}, \apj, {944}, 99 (\arxiv{2301.01694}) [\href{https://ui.adsabs.harvard.edu/abs/2023ApJ...944...99S}{9 citations}]

\item[{\color{numcolor}\scriptsize19}] Lillo-Box, J.; Gandolfi, D.; Armstrong, D. J.; Collins, K. A. \etal\ ({62} other co-authors, incl.\ \textbf{Savel, Arjun}) 2023, \doi{10.1051/0004-6361/202243879}{TOI-969: a late-K dwarf with a hot mini-Neptune in the desert and an eccentric cold Jupiter}, \aanda, {669} (\arxiv{2210.08996}) [\href{https://ui.adsabs.harvard.edu/abs/2023A&A...669A.109L}{8 citations}]

\item[{\color{numcolor}\scriptsize18}] \textbf{Savel}, \textbf{Arjun}; Hirsch, Lea A.; *Gill, Holden; Dressing, Courtney D. \etal 2022, \doi{10.1088/1538-3873/aca4f9}{SImMER: A Pipeline for Reducing and Analyzing Images of Stars}, \pasp, {134}, 124501 (\arxiv{2212.00641}) [\href{https://ui.adsabs.harvard.edu/abs/2022PASP..134l4501S}{3 citations}]

\item[{\color{numcolor}\scriptsize17}] Beltz, Hayley; Rauscher, Emily; Kempton, Eliza M. -R.; Malsky, Isaac \etal\ ({4} other co-authors, incl.\ \textbf{Savel, Arjun}) 2022, \doi{10.3847/1538-3881/ac897b}{Magnetic Drag and 3D Effects in Theoretical High-resolution Emission Spectra of Ultrahot Jupiters: the Case of WASP-76b}, \aj, {164}, 140 (\arxiv{2204.12996}) [\href{https://ui.adsabs.harvard.edu/abs/2022AJ....164..140B}{13 citations}]

\item[{\color{numcolor}\scriptsize16}] Esparza-Borges, E.; Parviainen, H.; Murgas, F.; Pall{\'e}, E. \etal\ ({45} other co-authors, incl.\ \textbf{Savel, Arjun}) 2022, \doi{10.1051/0004-6361/202243731}{A hot sub-Neptune in the desert and a temperate super-Earth around faint M dwarfs. Color validation of TOI-4479b and TOI-2081b}, \aanda, {666} (\arxiv{2206.10643}) [\href{https://ui.adsabs.harvard.edu/abs/2022A&A...666A..10E}{4 citations}]

\item[{\color{numcolor}\scriptsize15}] Newton, Elisabeth R.; Rampalli, Rayna; Kraus, Adam L.; Mann, Andrew W. \etal\ ({36} other co-authors, incl.\ \textbf{Savel, Arjun}) 2022, \doi{10.3847/1538-3881/ac8154}{TESS Hunt for Young and Maturing Exoplanets (THYME). VII. Membership, Rotation, and Lithium in the Young Cluster Group-X and a New Young Exoplanet}, \aj, {164}, 115 (\arxiv{2206.06254}) [\href{https://ui.adsabs.harvard.edu/abs/2022AJ....164..115N}{14 citations}]

\item[{\color{numcolor}\scriptsize14}] Gandhi, Siddharth; Kesseli, Aurora; Snellen, Ignas; Brogi, Matteo \etal\ ({5} other co-authors, incl.\ \textbf{Savel, Arjun}) 2022, \doi{10.1093/mnras/stac1744}{Spatially resolving the terminator: variation of Fe, temperature, and winds in WASP-76 b across planetary limbs and orbital phase}, \mnras, {515}, 749 (\arxiv{2206.11268}) [\href{https://ui.adsabs.harvard.edu/abs/2022MNRAS.515..749G}{12 citations}]

\item[{\color{numcolor}\scriptsize13}] Yee, Samuel W.; Winn, Joshua N.; Hartman, Joel D.; Rodriguez, Joseph E. \etal\ ({69} other co-authors, incl.\ \textbf{Savel, Arjun}) 2022, \doi{10.3847/1538-3881/ac73ff}{The TESS Grand Unified Hot Jupiter Survey. I. Ten TESS Planets}, \aj, {164}, 70 (\arxiv{2205.09728}) [\href{https://ui.adsabs.harvard.edu/abs/2022AJ....164...70Y}{11 citations}]

\item[{\color{numcolor}\scriptsize12}] Gan, Tianjun; Soubkiou, Abderahmane; Wang, Sharon X.; Benkhaldoun, Zouhair \etal\ ({63} other co-authors, incl.\ \textbf{Savel, Arjun}) 2022, \doi{10.1093/mnras/stac1448}{TESS discovery of a sub-Neptune orbiting a mid-M dwarf TOI-2136}, \mnras, {514}, 4120 (\arxiv{2202.10024}) [\href{https://ui.adsabs.harvard.edu/abs/2022MNRAS.514.4120G}{14 citations}]

\item[{\color{numcolor}\scriptsize11}] Murakami, Yukei S.; Jennings, Connor; Hoffman, Andrew M.; \textbf{Savel}, \textbf{Arjun} \etal\ ({7} other co-authors, incl.\ \textbf{Savel, Arjun}) 2022, \doi{10.1093/mnras/stac1538}{PIPS, an advanced platform for period detection in time series - I. Fourier-likelihood periodogram and application to RR Lyrae stars}, \mnras, {514}, 4489 (\arxiv{2107.14223}) [\href{https://ui.adsabs.harvard.edu/abs/2022MNRAS.514.4489M}{2 citations}]

\item[{\color{numcolor}\scriptsize10}] Giacalone, Steven; Dressing, Courtney D.; Hedges, Christina; Kostov, Veselin B. \etal\ ({108} other co-authors, incl.\ \textbf{Savel, Arjun}) 2022, \doi{10.3847/1538-3881/ac4334}{Validation of 13 Hot and Potentially Terrestrial TESS Planets}, \aj, {163}, 99 (\arxiv{2201.12661}) [\href{https://ui.adsabs.harvard.edu/abs/2022AJ....163...99G}{10 citations}]

\item[{\color{numcolor}\scriptsize9}] Dong, Jiayin; Huang, Chelsea X.; Zhou, George; Dawson, Rebekah I. \etal\ ({56} other co-authors, incl.\ \textbf{Savel, Arjun}) 2022, \doi{10.3847/2041-8213/ac4da0}{NEID Rossiter-McLaughlin Measurement of TOI-1268b: A Young Warm Saturn Aligned with Its Cool Host Star}, \apj, {926} (\arxiv{2201.12836}) [\href{https://ui.adsabs.harvard.edu/abs/2022ApJ...926L...7D}{12 citations}]

\item[{\color{numcolor}\scriptsize8}] \textbf{Savel}, \textbf{Arjun}; Kempton, Eliza M. -R.; Malik, Matej; Komacek, Thaddeus D. \etal 2022, \doi{10.3847/1538-4357/ac423f}{No Umbrella Needed: Confronting the Hypothesis of Iron Rain on WASP-76b with Post-processed General Circulation Models}, \apj, {926}, 85 (\arxiv{2109.00163}) [\href{https://ui.adsabs.harvard.edu/abs/2022ApJ...926...85S}{26 citations}]

\item[{\color{numcolor}\scriptsize7}] de Leon, J. P.; Livingston, J.; Endl, M.; Cochran, W. D. \etal\ ({24} other co-authors, incl.\ \textbf{Savel, Arjun}) 2021, \doi{10.1093/mnras/stab2305}{37 new validated planets in overlapping K2 campaigns}, \mnras, {508}, 195 (\arxiv{2108.05621}) [\href{https://ui.adsabs.harvard.edu/abs/2021MNRAS.508..195D}{17 citations}]

\item[{\color{numcolor}\scriptsize6}] May, Erin M.; Komacek, Thaddeus D.; Stevenson, Kevin B.; Kempton, Eliza M. -R. \etal\ ({15} other co-authors, incl.\ \textbf{Savel, Arjun}) 2021, \doi{10.3847/1538-3881/ac0e30}{Spitzer Phase-curve Observations and Circulation Models of the Inflated Ultrahot Jupiter WASP-76b}, \aj, {162}, 158 (\arxiv{2107.03349}) [\href{https://ui.adsabs.harvard.edu/abs/2021AJ....162..158M}{30 citations}]

\item[{\color{numcolor}\scriptsize5}] Cloutier, Ryan; Charbonneau, David; Stassun, Keivan G.; Murgas, Felipe \etal\ ({63} other co-authors, incl.\ \textbf{Savel, Arjun}) 2021, \doi{10.3847/1538-3881/ac0157}{TOI-1634 b: An Ultra-short-period Keystone Planet Sitting inside the M-dwarf Radius Valley}, \aj, {162}, 79 (\arxiv{2103.12790}) [\href{https://ui.adsabs.harvard.edu/abs/2021AJ....162...79C}{26 citations}]

\item[{\color{numcolor}\scriptsize4}] Foreman-Mackey, Daniel; Luger, Rodrigo; Agol, Eric; Barclay, Thomas \etal\ ({13} other co-authors, incl.\ \textbf{Savel, Arjun}) 2021, \doi{10.21105/joss.03285}{exoplanet: Gradient-based probabilistic inference for exoplanet data \& other astronomical time series}, JOSS, {6}, 3285 (\arxiv{2105.01994}) [\href{https://ui.adsabs.harvard.edu/abs/2021JOSS....6.3285F}{126 citations}]

\item[{\color{numcolor}\scriptsize3}] Rodriguez, Joseph E.; Quinn, Samuel N.; Zhou, George; Vanderburg, Andrew \etal\ ({115} other co-authors, incl.\ \textbf{Savel, Arjun}) 2021, \doi{10.3847/1538-3881/abe38a}{TESS Delivers Five New Hot Giant Planets Orbiting Bright Stars from the Full-frame Images}, \aj, {161}, 194 (\arxiv{2101.01726}) [\href{https://ui.adsabs.harvard.edu/abs/2021AJ....161..194R}{28 citations}]

\item[{\color{numcolor}\scriptsize2}] \textbf{Savel}, \textbf{Arjun}; Dressing, Courtney D.; Hirsch, Lea A.; Ciardi, David R. \etal 2020, \doi{10.3847/1538-3881/abc47d}{A Closer Look at Exoplanet Occurrence Rates: Considering the Multiplicity of Stars without Detected Planets}, \aj, {160}, 287 (\arxiv{2011.09564}) [\href{https://ui.adsabs.harvard.edu/abs/2020AJ....160..287S}{26 citations}]

\item[{\color{numcolor}\scriptsize1}] Demory, B. -O.; Pozuelos, F. J.; G{\'o}mez Maqueo Chew, Y.; Sabin, L. \etal\ ({70} other co-authors, incl.\ \textbf{Savel, Arjun}) 2020, \doi{10.1051/0004-6361/202038616}{A super-Earth and a sub-Neptune orbiting the bright, quiet M3 dwarf TOI-1266}, \aanda, {642} (\arxiv{2009.04317}) [\href{https://ui.adsabs.harvard.edu/abs/2020A&A...642A..49D}{56 citations}]
    \end{list}

    \cvheading{Preprints}
    \begin{list}{}{\cvlist}
    \item[{\color{numcolor}\scriptsize3}] \textbf{Savel, Arjun}; Kempton, Eliza M. -R.; Malik, Matej; Komacek, Thaddeus D.; \etal, 2021, \emph{No umbrella needed: Confronting the hypothesis of iron rain on WASP-76b with post-processed general circulation models}, ArXiv (\arxiv{2109.00163})

\item[{\color{numcolor}\scriptsize2}] May, Erin M.; Komacek, Thaddeus D.; Stevenson, Kevin B.; Kempton, Eliza M. -R.; \etal\ ({14} other co-authors, incl.\ \textbf{Savel, Arjun}), 2021, \emph{Spitzer phase curve observations and circulation models of the inflated ultra-hot Jupiter WASP-76b}, ArXiv (\arxiv{2107.03349}) [\href{https://ui.adsabs.harvard.edu/abs/2021arXiv210703349M}{1 citation}]

\item[{\color{numcolor}\scriptsize1}] Murakami, Yukei S.; Jennings, Connor; Hoffman, Andrew M.; Sunseri, James; \etal\ ({6} other co-authors, incl.\ \textbf{Savel, Arjun}), 2021, \emph{PIPS, an advanced platform for period detection in time series -- I. Fourier-likelihood periodogram and application to RR Lyrae Stars}, ArXiv (\arxiv{2107.14223})
    \end{list}
\fi

\cvheading{Science Talks and Posters}
\begin{list}{}{\cvlist}

\item 8. Courtney D. Dressing, Steven Giacalone, Ellianna S. Abrahams \& 7 coauthors including \textbf{Arjun Savel}, 2020. ``Using TESS to Investigate the Frequency of Planetary Systems Orbiting Cool Dwarfs'', AAS 235, Honolulu, Hawai'i

\item 7. $^\ddagger$\textbf{Arjun Savel}, Courtney D. Dressing, Lea Hirsch, David Ciardi, Jordan P.C. Fleming, Steven Giacalone, Andrew W. Mayo, Jessie L. Christiansen, 2020. “A closer look at planet occurrence rates: AO follow-up of 71 stars in the Kepler field”, AAS 235, Honolulu, Hawai'i

\item 6. \textbf{Arjun Savel}, Courtney D. Dressing, Lea Hirsch, David Ciardi, Jordan P.C. Fleming, Steven Giacalone, Andrew W. Mayo, Jessie L. Christiansen, 2019. “A Closer Look at Exoplanet Occurrence Rates: Considering the Multiplicity of Stars without Detected Planets”, Bay Area Exoplanet Meeting \#31, NASA Ames

\item 5. \textbf{Arjun Savel}, Courtney D. Dressing, Lea Hirsch, David Ciardi, Jordan P.C. Fleming, Steven Giacalone, Andrew W. Mayo, Jessie L. Christiansen, 2019. “A Closer Look at Exoplanet Occurrence Rates: The Impact of Stars Without Exoplanets”, Bay Area Planetary Sciences Meeting, Stanford University.

\item 4. \textbf{Arjun Savel}, 2019.“Earth: Rare or Regular?”, Undergraduate Seminars, UC Berkeley Physics Department.

\item 3. $^*$\textbf{Arjun Savel}, Courtney D. Dressing, Lea Hirsch, David Ciardi, Jordan P. C. Fleming, Jessie L. Christiansen, 2019. “A closer look: AO follow-up of 109 stars in the Kepler and K2 fields”, APSIS Poster Session, UC Berkeley.

\item 2. Courtney D. Dressing, \textbf{Arjun Savel} et al. 2019. “Characterizing Planetary Systems Orbiting TESS Cool Dwarfs”, TESS Science Conference I, MIT.

\item 1. Steven Giacalone, Courtney Dressing, \textbf{Arjun Savel}, 2019. “Validation of TESS Exoplanet Candidates”, 3rd Advanced School on Exoplanetary Science, Vietri sul Mare.

\end{list}

\cvheading{Public Talks}
\begin{list}{}{\cvlist}
\item 1. Courtney Dressing, Steven Giacalone, Andrew W. Mayo, \textbf{Arjun Savel}. Evening with the Stars, UC Berkeley, 2020
\end{list}



\cvheading{Observing Experience}
\begin{list}{}{\cvlist}

\item 3-m Shane telescope (ShARCS), Mt. Hamilton, CA: assisted with 10 nights
\item 10-meter Keck Telescope (NIRC2), Mauna Kea, HI: assisted with 1/2 night
\item 10-meter Keck Telescope (NIRSPEC), Mauna Kea, HI: assisted with 1/2 night

\end{list}

% \ifdefined\withpubs
%     \newpage
% \fi

\cvheading{Teaching Experience}
\begin{list}{}{\cvlist}
\item Undergraduate Student Instructor, Astronomy C12 (The Planets), UC Berkeley, under Courtney Dressing and Raymond Jeanloz (2020)
\item $^\dagger$Undergraduate Student Instructor, Astronomy C10 (Introduction to General Astronomy), UC Berkeley, under Alex Filippenko (2018-19)
\end{list}

\cvheading{Community Involvement}
\begin{list}{}{\cvlist}
\item Mentor, TARDIS Google Summer of Code (2020)
\item Undergraduate Representative, Astronomy Department, University of California, Berkeley (2019-20)
\item Director, Undergraduate Lab at Berkeley (ULAB), University of California, Berkeley (2018-19)
\end{list}

\cvheading{Workshops, Conference Participation, and Other Experience}
\begin{list}{}{\cvlist}
\item JWST Master Class Workshop, Stanford University (2020)
\item Bay Area Exoplanet Meeting, NASA Ames (Spring 2019, Winter 2019, Spring 2020)
\item Bay Area Planetary Science Meeting, Stanford University (2019)
\item Night Editor, The Daily Californian (2017)
\end{list}

\cvheading{Technical Skills}
\begin{list}{}{\cvlist}
\item Python, ADQL/SQL, R, MCMC, neural networks, astronomical image reduction, open-source code management
\end{list}

\end{document}